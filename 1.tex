\documentclass[10pt,a4paper]{article}
\usepackage[utf8]{inputenc}
\usepackage{amsmath}
\usepackage{amsfonts}
\usepackage{amssymb}
\begin{document}
\section{Formas Canónicas}
\subsection{Valores y Vectores Propios }
En lo que sigue de este capitulo, $U$, $V$, $W$ denoatarán $\mathbb{K}-$espacios vectoriales de dimención finita y $\mathbb{K}$ denotará los cuerpos $\mathbb{R}$ o $\mathbb{C}$, salvo mención especifica distinta.\\
\textbf{Definición.} Dada una transformación lineal $T : V \rightarrow{V},$ un número $\lambda \in \mathbb{K}$ se llama \textbf{valor propio} o \textbf{autovalor} de $T$, si existe un vector $v \in V,$ $v \neq 0,$ tal que $T(v) = \lambda v.$ Este vector se llama \textbf{vector propio} o \textbf{autovector} de $T$ correspondiente al autovalor $\lambda$.

Llamaremos valor propio y vector propio de una matriz $A$, al valor propio y vector propio correspondiente de la transformación lineal $L_{A}$, respectivamente. \\
Los autovectores de $T$ y $A_{T}$, en general, se hallan en espacios vectoriales distintos y no tienen que ser iguales. En cambio los autovalores se hallan en el mismo cuerpo $\mathbb{K}$, por lo que cabe la pregunta ¿son éstos iguales o distintos? Una elegante respuesta a esta interrogante se da en las siguientes proposiciones.\\

Sea $\{v_{1},...v_{n}\}$ una base de $V$ y $A_{T}$ la matriz asociada a la transformación lineal $T: V \rightarrow{V}$ en esta base, A cada vector $v \in V$ , $v = \displaystyle\sum_{j=1}^{n}x_{j}v_{j},$ le asociamos su vector de coordenadas $x_{v} = (x_{1},...,x_{n}) \in \mathbb{K}$. Con estas notaciones tenemos el siguiente resultado.\\

\textbf{Proposición.} La función $\psi : V \rightarrow \mathbb{K}^{n \times 1}$, definida por $\psi(v) = x_{v}$, es un isomorfismo y satisface 
$$\psi(T(v)) = A_{T}(x_{v})$$






\end{document}