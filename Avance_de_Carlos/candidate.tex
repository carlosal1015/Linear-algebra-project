\documentclass{article}
\begin{document}
\section{Speeding up finite-time consensus via minimal polynomial of a weighted graph\\[\baselineskip]
Acelerar el consenso de tiempo finito a través de un polinomio mínimo de un gráfico ponderado}

%https://www.sciencedirect.com/science/article/pii/S0005109818301626?via%3Dihub
\begin{abstract}

This work proposes an approach to speed up finite-time consensus algorithm using the weights of a weighted Laplacian matrix. It is motivated by the need to reach consensus among states of a multi-agent system in a distributed control/optimization setting. The approach is an iterative procedure that finds a low-order minimal polynomial that is consistent with the topology of the underlying graph. In general, the lowest-order minimal polynomial achievable for a network system is an open research problem. This work proposes a numerical approach that searches for the lowest order minimal polynomial via a rank minimization problem using a two-step approach: the first being an optimization problem involving the nuclear norm and the second a correction step. Convergence of the algorithm is shown and effectiveness of the approach is demonstrated via several examples.\\[\baselineskip]

Este trabajo propone un enfoque para acelerar el algoritmo de consenso de tiempo finito utilizando los pesos de una matriz laplaciana ponderada. Está motivado por la necesidad de llegar a un consenso entre los estados de un sistema multiagente en una configuración de control / optimización distribuida. El enfoque es un procedimiento iterativo que encuentra un polinomio mínimo de bajo orden que es consistente con la topología del gráfico subyacente. En general, el polinomio mínimo de orden inferior alcanzable para un sistema de red es un problema de investigación abierto. Este trabajo propone un enfoque numérico que busca el polinomio mínimo de orden más bajo a través de un problema de minimización de rango utilizando un enfoque de dos pasos: el primero es un problema de optimización que involucra la norma nuclear y el segundo es un paso de corrección. Se muestra la convergencia del algoritmo y se demuestra la efectividad del enfoque a través de varios ejemplos.

\end{abstract}

Achieving consensus of states is a well-known important feature for networked system, see for example Olfati-Saber and Murray (2004) and Ren and Beard (2007). Many distributed con- trol/optimization problems over a network require a consensus algorithm as a key component. The most common consensus al- gorithm is the dynamical system defined by the Laplacian matrix for continuous time system and the Perron matrix for discrete-time system. Past works in the general direction of speeding up conver- gence of these algorithms exist. For example, the work of Xiao and Boyd (2004) proposes a semi-definite programming approach to minimize the algebraic connectivity over the family of symmetric matrices that are consistent with the topology of the network. Their approach, however, results in asymptotic convergence to- wards the consensus value and is most suitable for large networks. More recent works focus on finite-time convergence consensus al- gorithm (Hendrickx, Jungers, Olshevsky, \& Vankeerberghen, 2014; Hendrickx, Shi, \& Johansson, 2015; Sundaram \& Hadjicostis, 2007; Wang \& Xiao, 2010; Yuan, Stan, Shi, Barahona, \& Goncalves, 2013; Yuan, Stan, Shi, \& Goncalves, 2009) which is generally preferred for small to moderate sized networks. One important area in finite-time convergence literature is the determination of the determination of the asymptotic value of a consensus network using a finite number of state measurement. Typically, the approach adopted is based on the z-transform final-value theorem and on the finite-time convergence for individual node (Sundaram \& Hadjicostis, 2007; Yuan et al., 2013, 2009) without knowledge of the full network.

Other works in finite-time consensus include the design of a
short sequence of stochastic matrices Ak , . . . , A0 such that z (k) =
Π k Aj z (0) reaches consensus after k steps (Hendrickx et al., 2015; j=1
Ko \& Shi, 2009).
Unlike past works (Sundaram \& Hadjicostis, 2007; Yuan et al.,
2013, 2009) where the network is unknown, this work assumed a known network and proposes an approach to speed up finite-time convergence of consensus algorithm via the choices of the weights associated with the edges of the graph. Thus, it is similar in spirit to the work of Xiao and Boyd (2004) except that the intention is to find a low-order minimal polynomial. Ideally, the lowest-order minimal polynomial should be used. However, the lowest minimal polynomial achievable for a given graph with variable weights is an open research problem (Fallat \& Hogben, 2007). They are only known for some special classes of graphs (full connected, star- shaped, strongly regular and others), van Dam and Haemers (1998) and van Dam, Koolen, and Tanaka (2014). For this reason, this paper adopts a computational approach towards finding a low-order minimal polynomial. The proposed approach achieves the lowest order minimal polynomial in many of the special classes of graphs and almost always yields minimal polynomial of order lower than those obtained from standard Perron matrices of general graphs. These are demonstrated by several numerical examples.

\newpage

Lograr el consenso de los estados es una característica importante bien conocida para el sistema en red, ver, por ejemplo, Olfati-Saber y Murray (2004) y Ren y Beard (2007). Muchos problemas de control / optimización distribuidos en una red requieren un algoritmo de consenso como componente clave. El algoritmo de consenso más común es el sistema dinámico definido por la matriz laplaciana para el sistema de tiempo continuo y la matriz de Perron para el sistema de tiempo discreto. Los trabajos anteriores en la dirección general de acelerar la convergencia de estos algoritmos existen. Por ejemplo, el trabajo de Xiao y Boyd (2004) propone un enfoque de programación semi-definido para minimizar la conectividad algebraica sobre la familia de matrices simétricas que son consistentes con la topología de la red. Sin embargo, su enfoque resulta en una convergencia asintótica hacia el valor de consenso y es más adecuado para redes grandes. Los trabajos más recientes se centran en el algoritmo de consenso de convergencia de tiempo finito (Hendrickx, Jungers, Olshevsky, \& Vankeerberghen, 2014; Hendrickx, Shi, \& Johansson, 2015; Sundaram \& Hadjicostis, 2007; Wang \& Xiao, 2010; Yuan, Stan, Shi, Barahona y Goncalves, 2013; Yuan, Stan, Shi y Goncalves, 2009) que generalmente se prefiere para redes de tamaño pequeño a moderado. Un área importante en la literatura de convergencia en tiempo finito es la determinación de la determinación del valor asintótico de una red de consenso utilizando un número finito de medición de estado. Típicamente, el enfoque adoptado se basa
en el teorema del valor final de la transformada zy en el tiempo finito
convergencia para nodo individual (Sundaram y Hadjicostis, 2007;
Yuan et al., 2013, 2009) sin conocimiento de la red completa.
Otros trabajos en consenso de tiempo finito incluyen el diseño de un
secuencia corta de matrices estocásticas Ak,. . . , A0 tal que z (k) =
Π k Aj z (0) alcanza el consenso después de k pasos (Hendrickx et al., 2015; j = 1
Ko \& Shi, 2009).
A diferencia de trabajos anteriores (Sundaram y Hadjicostis, 2007; Yuan et al.,
2013, 2009) donde la red es desconocida, este trabajo asumió una red conocida y propone un enfoque para acelerar la convergencia de tiempo finito del algoritmo de consenso a través de las opciones de los pesos asociados con los bordes del gráfico. Por lo tanto, es similar en espíritu al trabajo de Xiao y Boyd (2004), excepto que la intención es encontrar un polinomio mínimo de orden inferior. Idealmente, debería usarse el polinomio mínimo de orden más bajo. Sin embargo, el polinomio mínimo más bajo posible para un gráfico dado con pesos variables es un problema de investigación abierta (Fallat y Hogben, 2007). Solo son conocidos por algunas clases especiales de gráficos (totalmente conectados, en forma de estrella, muy regulares y otros), van Dam and Haemers (1998) y van Dam, Koolen y Tanaka (2014). Por esta razón, este documento adopta un enfoque computacional para encontrar un polinomio mínimo de bajo orden. El enfoque propuesto logra el polinomio mínimo de orden más bajo en muchas de las clases especiales de gráficos y casi siempre produce un polinomio mínimo de orden inferior a los obtenidos de las matrices de Perron estándar de los gráficos generales. Estos son demostrados por varios ejemplos numéricos.

\end{document}